\section{Zusammenfassung}

\begin{frame}{Kurz \& Knapp}    
    \begin{itemize}
        \item Schnelle Generierung von großen Datenmengen
        \item<2-> Beibehaltung aller Eigenschaften der ursprünglichen Funktion
        \item<3-> einfaches Prinzip
    \end{itemize}
\end{frame}

\begin{frame}{Verbesserungsmöglichkeit}   
    \begin{itemize}
        \item\textbf{Problem:} Funktion mit sehr hoher Dichte an wenigen Stellen, sonst nur geringe Dichte.
        \begin{itemize}[label=$\rightarrow$]
            \item<2-> Dort fallen alle Punkte zusammen.
            \item<3-> Hohe Ungenauigkeit an wichtigen Stellen.
        \end{itemize}
        \item<4->\textbf{Lösung:} Hashtabelle nicht gleichmäßig populieren, sondern für diese 
        Stellen höhere Genauigkeit durch mehr Einträge ermöglichen.
        \item<5->\textbf{Allerdings:} stark erhöhter Initialisierungsaufwand
    \end{itemize} 
\end{frame}