\documentclass{sdqbeamer}
% Handout-Mode
%\documentclass[handout]{sdqbeamer}
\makeatletter
\newcommand*{\inlineequation}[2][]{%
  \begingroup
    % Put \refstepcounter at the beginning, because
    % package `hyperref' sets the anchor here.
    \refstepcounter{equation}%
    \ifx\\#1\\%
    \else
      \label{#1}%
    \fi
    % prevent line breaks inside equation
    \relpenalty=10000 %
    \binoppenalty=10000 %
    \ensuremath{%
      % \displaystyle % larger fractions, ...
      #2%
    }%
    ~\@eqnnum
  \endgroup
}
\makeatother

% Ein paar hilfreiche Pakete
\usepackage[utf8]{inputenc}
\usepackage[ngerman]{babel}
\usepackage{array}
\usepackage{enumitem}
\usepackage{graphicx}
\usepackage{multirow}
\usepackage{tabularray}
\usepackage{color}
\usepackage{colortbl}
\usepackage{caption}
\usepackage{subcaption}
\usepackage{amsmath}
\usepackage{amssymb}
\usepackage{hyperref}
\usepackage{algpseudocode}
\usepackage{algorithm}
 
% Für Code-Schnipsel
\def\code#1{\texttt{#1}}

% Für Pseudocode input und output einer funktion
\algblock{Input}{EndInput}
\algnotext{EndInput}
\algblock{Output}{EndOutput}
\algnotext{EndOutput}
\newcommand{\Desc}[2]{\State \makebox[3em][l]{#1}#2}

% Graphics Paths
\graphicspath{{../../Media}}

% Für subfigure plots
\captionsetup[subfigure]{justification=centering}

% Für "normale" itemize
\setlist[itemize,1]{label={\raisebox{.2ex}{\KITmark}}}
\setlist[itemize,2]{label={\raisebox{.2ex}{\KITmark}}}

% Für gefärbte Tabellen
\renewcommand<>\cellcolor[1]{\only#2{\beameroriginal\cellcolor{#1}}}
\definecolor{pyblue}{rgb}{0.361, 0.725, 1.0}
\definecolor{pyorange}{rgb}{1.0, 0.725, 0.285}
\definecolor{Gray}{gray}{0.9}
\newcolumntype{G}{>{\columncolor{Gray}}c} % colored column

%% Titelbild
\titleimage{banner_2020_kit}

%% Gruppenlogo

%\grouplogo{ISAS_logo} 

% Beginn der Präsentation

\title[Punkt-Warping]{Punkt-Warping: Inversionsmethode mit Hashing}
%\subtitle{entsprechend den Gestaltungsrichtlinien vom 1. August 2020} 
\author[Lukas Knirsch]{Lukas Knirsch}
\supervisor{Daniel Frisch}

\date[12.\,01.\,2022]{12. Januar 2022}

% Literatur 
 
\usepackage[citestyle=authoryear,bibstyle=numeric,hyperref,backend=biber]{biblatex}
\addbibresource{presentation.bib}
\bibhang1em

\begin{document}

% Titelseite
\KITtitleframe

% Inhaltsverzeichnis
\begin{frame}{Inhaltsverzeichnis}
	\tableofcontents
\end{frame}

% Text
\section{Einstieg}

\begin{frame}{Problemstellung}
	\begin{itemize}
		\item Punkte nach Dichtefunktion im Raum verteilen
		\item<2-> Mehrere existierende Verfahren
		\begin{itemize}
			\item Halton (\cite{wong_luk_heng-halton_points_sampling-1997})
			\item Blue Noise (\cite{yan-blue_noise_sampling-2015})
		\end{itemize}
		\item<3-> \textit{Erweiterung:} gleichmäßige Verteilung
		\begin{itemize}
			\item Golden Ratio Sequences + Halton/Blue Noise (\cite{schretter-golden_ratio_sequences-2012})
			\item Fibonacci Grids (\cite{frisch_hanebeck-deterministic_gaussian_sampling-2021})
		\end{itemize}
		\item<4->\textbf{Ziel 1:} effizientes Vorgehen 
		\item<5->\textbf{Ziel 2:} mehrdimensional anwendbar
		\item<6->\textbf{Ziel 3:} unabhängig von bestimmten Eigenschaften der Funktion einsetzbar
	\end{itemize}
\end{frame}

\begin{frame}{Ziel}
	\begin{figure}
		\begin{subfigure}{.45\textwidth}
			\centering
			\includegraphics[width=\textwidth]{pdf_plots/grs_u500c.pdf}
		\end{subfigure}
		% \hfill
		% \begin{subfigure}{.15\textwidth}
		% 	\centering
		% 	\includegraphics[width=.5\textwidth, angle=180]{Screenshots/img_71815.png}
		% \end{subfigure}
		% \hfill
		\begin{subfigure}{.45\textwidth}
			\centering
			\includegraphics[width=\textwidth]{pdf_plots/idf2_log-log_iu500c-GRS.pdf}
		\end{subfigure}
		\caption{\cite{schretter-golden_ratio_sequences-2012}}
	\end{figure}
\end{frame}
\section{Funktionsweise}

\begin{frame}{Inversionsmethode}
	\begin{figure}
		\begin{subfigure}{.45\textwidth}
			\centering
			\includegraphics[width=\textwidth]{pdf_plots/logistic_-4to4.pdf}
            \caption{F mit Werten von $-4$ bis $4$}
		\end{subfigure}
		\begin{subfigure}{.45\textwidth}
			\centering
			\includegraphics[width=\textwidth]{pdf_plots/logistic_inv_0to1.pdf}
            \caption{F$^{-1}$ mit Werten von $0$ bis $1$}
		\end{subfigure}
	\end{figure}
\end{frame}

\begin{frame}{Inversionsmethode}
    \begin{itemize}
        \item Inverse einer Dichtefunktion benötigt 
        \item Entweder bekannt %\onslide<2->%, siehe \textit{\hyperref[tab:invFunctions]{Tabelle}} 
            oder numerisch integrierbar/annäherbar
    \end{itemize}
    \begin{table}%[htb!]
        \centering
        \begin{tabular}{l|l|l}
            Name         & Funktion & Zufällige Variable \\
            \hline\hline %& & \\
            Exponentiell & $1 - e^{-x}$ & $\log(1/U)$ \\ %\onslide<2->
            Logistisch   & $1 / (1 + e^{-x})$ & $-\log(\dfrac{1-U}{U})$ \\ %\onslide<3->
            Cauchy       & $1/2 + (1/\pi) \arctan(x)$ & $\tan(\pi U)$
        \end{tabular}
        \label{tab:invFunctions}
    \end{table}
\end{frame}

\begin{frame}{Hash-basierte Inversionsmethode}
    \begin{table}%[htb!]
        \centering
        \begin{tabular}{l|l|l|l}
            $X_j$   & F$(X_j)$  & $I_j$ & T$(I_j)$ \\
            \hline\hline % & & & \\
            $v_1$   & $0.021$   & $1$   & 1 \\ 
            $v_2$   & $0.021$   & $1$   & 1 \\
            $v_3$   & $0.021$   & $1$   & 1 \\
            $v_4$   & $0.021$   & $1$   & 1 \\ 
            $v_5$   & $0.021$   & $1$   & 1 \\
            $v_6$   & $0.021$   & $1$   & 1
        \end{tabular}
        \hspace{2em}
        \begin{tabular}{l|l|l|l}
            $X_j$   & F$(X_j)$  & $I_j$ & T$(I_j)$ \\
            \hline\hline % & & & \\
            $v_7$   & $0.021$   & $1$   & 1 \\ 
            $v_8$   & $0.021$   & $1$   & 1 \\
            $v_9$   & $0.021$   & $1$   & 1 \\
            $v_{10}$  & $0.021$   & $1$   & 1 \\ 
            $v_{11}$  & $0.021$   & $1$   & 1 \\
            $v_{12}$  & $0.021$   & $1$   & 1
        \end{tabular}
    \end{table}
\end{frame}
\section{Vergleich}

\begin{frame}{Vergleich der Invertierungsmethoden}
    \begin{figure}
        \centering
        \includegraphics[width=.7\textwidth]{Screenshots/chenAsauHashTableComp.pdf}
        \caption{\cite{chen_asau-generating_random_variates-1974}}
    \end{figure}
\end{frame}


\section{Implementierung}
\label{impl}
\begin{figure*}[htb!]
    \centering
    \begin{subfigure}[b]{.3\textwidth}
        \centering
        \includegraphics[width=\textwidth]{pdf_plots/idf2_log-sin_iu500c-RND.pdf}
        \caption{Eingabewerte zufällig generiert mit \code{random}}
        \label{fig:logsin_random}
    \end{subfigure}
    \hfill
    \begin{subfigure}[b]{.3\textwidth}
        \centering
        \includegraphics[width=\textwidth]{pdf_plots/idf2_log-sin_iu500c-GRS.pdf}
        \caption{Eingabewerte aus einer Goldenen Schnitt Sequenz}
        \label{fig:logsin_uniform}
    \end{subfigure}
    \caption{$500$ Punkte mit einer logistischen Dichte auf der X- und einer sinusoiden auf der Y-Achse}
    \label{fig:rand_vs_uniform}
\end{figure*}


Um einen Eindruck der hashbasierten Inversionsmethode und ihrer Arbeitsweise zu bekommen, 
wird jetzt eine konkrete Implementierung dieser angesprochen und Ergebnisse bei 
unterschiedlichen Eingaben gezeigt. Geschrieben wurde das Programm in Python mithilfe der 
Pakete \textit{numpy} und \textit{matplotlib}.

Schretter \cite{schretter-golden_ratio_sequences-2012} stellt am Ende der Arbeit eine 
C++-Methode zur Generierung einer zweidimensionalen Goldenen-Schnitt-Sequenz zur Verfügung. 
Diese Methode wurde in ein Python-Script übertragen und alle, in diesem Kapitel erwähnte, zufällige, uniformen
Variablen $U$ werden mit diesem Script generiert.

Zum Testen der vorgestellten Methode wurde das Konzept von Chen und Asau \cite{chen_asau-generating_random_variates-1974} 
implementiert. Dabei wurde kein großer Wert auf optimale Laufzeiten gelegt, sondern darauf, eine Demonstration 
des Verfahrens geben zu können.

% TODO
% Ergebnisse bei unterschiedlichen Werten auswerten nicht nur im Bild
\dots



\section{Zusammenfassung}

\begin{frame}{Kurz \& Knapp}    
    \begin{itemize}
        \item Schnelle Generierung von großen Datenmengen
        \item Beibehaltung aller Eigenschaften der ursprünglichen Funktion
        \item einfaches Prinzip
    \end{itemize}
\end{frame}

\begin{frame}{Ausblick}   

\end{frame}

% Appendix
\appendix
\beginbackup

\section{Anhang}

\begin{frame}[allowframebreaks]{Literatur}
	\printbibliography
\end{frame}

\begin{frame}[plain]
	\centering\Huge Vielen Dank für Ihre Aufmerksamkeit!
\end{frame}

\backupend

\end{document}