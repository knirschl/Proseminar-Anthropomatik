\section{Einschränkungen}
Nachdem die Funktionsweise der Hash-basierten Inversionsmethode erklärt wurde, 
werden nun mögliche Einschränkungen der Methode näher betrachtet.


\subsection{Dimension}
Da in dem Abschnitt über die Funktionsweise der \hyperref[funktion]{Inversionsmethode} 
schon erklärt wurde, wie die Inversionsmethode in mehreren Dimensionen eingesetzt werden kann, 
ist leicht zu sehen, dass mit etwas Mehraufwand auch die hash-basierte Methode in n Dimensionen 
verwendet werden kann. Dieser Mehraufwand besteht hauptsächlich darin, nicht 
mehr nur eine Hashtabelle zu verwalten, sondern mehrere. 


\subsection{Abbildung}
Da die Inversionsmethode nur sinnvoll einsetzbar ist, wenn es möglich ist, das Inverse einer Funktion zu 
berechnen, oder die Dichtefunktion direkt bekannt ist, ist auch die hashbasierte Inversion nur dann einsetzbar.
Ansonsten erhält die Methode alle Eigenschaften ihrer Eingabefunktion. Dazu gehören Winkelerhalt, aber auch 
Korrelation und statistische Unabhängigkeit, da durch die Hashtabelle keine Eigenschaften verloren gehen, solange 
die Punkte nicht zu dicht beieinander liegen (siehe \hyperref[Dichte]{Dichte}).  


\subsection{Dichte}
\label{Dichte}
Da durch den Einsatz der Hashtabelle nur endlich viele Werte gespeichert werden können, ist eine Berechnung 
von Zufallsvariablen $X$ nur dann sinnvoll, wenn diese nicht zu dicht beieinander liegen und durch Annäherung 
auf einen Punkt gemappt werden würden. Deshalb sollte vor Einsatz der Tabelle die Dichte bekannt sein, damit die 
Größe $N$ der Hashtabelle optimal gewählt werden kann.

Unterscheidet sich die Dichte sehr stark an unterschiedlichen Punkten, kann sogar die Werteverteilung in der Tabelle 
angepasst werden, indem nicht mehr gleichmäßig verteilte Punkte eingesetzt werden, sondern je nach Dichte mehr 
oder weniger. Diese Anpassungen können die Initialisierung der Hashtabelle allerdings sehr viel langsamer machen, 
weshalb in solchen Fällen abgewogen werden muss, was die Hashtabelle können muss.