\section{Einschränkungen}
Nachdem die Funktionsweise der Hash-basierten Inversionsmethode erklärt wurde, 
werden nun mögliche Einschränkungen der Methode näher betrachtet und, wenn möglich, 
Lösungsansätze  aufgezeigt.


\subsection{Dimension}
Da in dem Abschnitt über die Funktionsweise der \hyperref[funktion]{Inversionsmethode} 
schon erklärt wurde, wie die Inversionsmethode in mehreren Dimensionen eingesetzt werden kann, 
ist leicht zu erkennen, dass mit etwas Mehraufwand auch die hash-basierte Methode in $n$ 
Dimensionen verwendet werden kann. Da die Werte der einzelnen Dimensionen rekursiv und 
abhängig voneinander berechnet werden, reicht nicht mehr nur eine Hashtabelle aus. Für 
jede Dimension muss eine Hashtabelle erstellt werden. Aufgrund der Abhängigkeit muss die 
Hashtabelle der höchsten Dimension zuerst berechnet werden. Dies funktioniert nach dem 
bekannten Muster, aber für alle weiteren Dimensionen müssen nun andere Funktionen zum 
Populieren der Tabellen verwendet werden. 

Anstelle der gegebenen Dichtefunktion $F$ mit einer Verteilung $\mathbb{P}(X) = p_X$ ist die 
jeweilige Verteilung gegeben durch $\mathbb{P}(X|Y)=\dfrac{\mathbb{P}(X\cap Y)}{\mathbb{P}(Y)}$, 
wobei $X$ der gesuchte Wert in der aktuellen Dimension ist und Y der Wert der höheren Dimension. 
An diesem Aufbau sieht man, dass die Hash-basierte Inversionsmethode zwar theoretisch für 
Punkt-Warping in beliebig vielen Dimensionen verwendet werden kann, doch bei zu vielen Dimensionen 
wird der Verwaltungsaufwand zu groß und die Initialisierung zu rechenaufwändig. Ab wievielen 
Dimensionen das der Fall ist, kann alllerdings nicht pauschal gesagt werden, da man mit kleineren 
Hashtabellen zwar ungenauer invertiert, aber weniger Rechenaufwand hat. 


\subsection{Abbildung}
Da die Inversionsmethode nur sinnvoll einsetzbar ist, wenn es möglich ist, das Inverse einer 
Funktion zu berechnen, oder die Dichtefunktion direkt bekannt ist, ist auch die hashbasierte 
Inversion nur dann einsetzbar. Ansonsten erhält die Methode alle Eigenschaften ihrer 
Eingabefunktion. Dazu gehören Winkelerhalt, aber auch Korrelation und statistische 
Unabhängigkeit, da durch die Hashtabelle keine Eigenschaften verloren gehen, solange 
die Punkte nicht zu dicht beieinander liegen (siehe \hyperref[Dichte]{Dichte}).


\subsection{Dichte}
\label{Dichte}
Da durch den Einsatz der Hashtabelle nur endlich viele Werte gespeichert werden können, ist 
eine Berechnung von Zufallsvariablen $X$ nur dann sinnvoll, wenn diese nicht zu dicht 
beieinander liegen und durch Annäherung auf einen Punkt gemappt werden würden, da die Tabelle 
eine zu geringe Größe hat, um ausreichend viele unterschiedliche Werte zu speichern. Deshalb 
sollte vor Einsatz der Tabelle die Dichte bekannt sein, damit die Größe $N$ der 
Hashtabelle optimal gewählt werden kann.

Unterscheidet sich die Dichte sehr stark an unterschiedlichen Punkten, kann sogar die 
Werteverteilung in der Tabelle angepasst werden, indem nicht mehr gleichmäßig verteilte 
Punkte eingesetzt werden, sondern je nach Dichte mehr oder weniger. Diese Anpassungen 
können die Initialisierung der Hashtabelle allerdings sehr viel langsamer oder größer machen, 
weshalb in solchen Fällen abgewogen werden muss, was die Hashtabelle können muss.