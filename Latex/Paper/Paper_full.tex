\documentclass[a4paper]{IEEEtran}

% Ein paar hilfreiche Pakete
\usepackage[utf8]{inputenc}
\usepackage[ngerman]{babel}
\usepackage{graphicx}
\usepackage{caption}
\usepackage{subcaption}
\usepackage{amsmath}
\usepackage{amssymb}
\usepackage{mathtools}
%\usepackage[section]{placeins}
\usepackage{hyperref}
\usepackage{algpseudocode}
\usepackage{algorithm}

% Nummeriere Formel nur, wenn sie auch referenziert wird
\mathtoolsset{showonlyrefs}

% Für Code-Schnipsel
\def\code#1{\texttt{#1}}

% Für Pseudocode input und output einer funktion
\algblock{Input}{EndInput}
\algnotext{EndInput}
\algblock{Output}{EndOutput}
\algnotext{EndOutput}
\newcommand{\Desc}[2]{\State \makebox[3em][l]{#1}#2}

% Graphics Paths
\graphicspath{{../../Media}}

% Für subfigure plots
\captionsetup[subfigure]{justification=centering}

% Header
\markboth{Proseminar Anthropomatik WS 21/22: Von der Theorie zur Anwendung}{Proseminar Anthropomatik WS 21/22: Von der Theorie zur Anwendung}

% Hier den Titel des eigenen Seminars eintragen
\title{Punkt-Warping: Inversionsmethode mit Hash-Tabelle}

% Hier deinen eigenen Namen
\author{Lukas Knirsch}

% --- START OF DOCUMENT ---
\begin{document}

% Erzeugt die Überschrift
\maketitle

% Abstract
\begin{abstract}
Bei wissenschaftlichen Simulationen oder bei hochqualitativen Schätzberechnungen werden sehr große Datensätze benötigt, 
die nach bestimmten Merkmalen verteilt sind \cite{frisch_hanebeck-deterministic_gaussian_sampling-2021}. Diese Daten 
können häufig durch Funktionen wie eine Normalverteilung oder durch multivariante Gaussche Verteilungen beschrieben werden. 
Umso gleichmäßiger die Daten aber verteilt sind, umso genauer können die verwendeten Algorithmen die Simulationen oder 
Schätzungen berechnen. Allerdings besitzen die dafür bereits verwendeten Methoden alle aber verschiedene Nachteile. Einer 
der häufigsten Nachteile ist der hohe Rechenaufwand für die ungefähr gleichmäßige Verteilung von Proben im Raum. 

Aus diesem Grund soll in dieser Ausarbeitung die Funktionsweise und eine Implementierung der hash-basierten 
Inversionsmethode erarbeitet werden und ihr möglicher Einsatz bei der gleichmäßigen Verteilung von Daten und Punkt-Warping 
im Allgemeinen.
\end{abstract}

% Text
\section{Einleitung}

In dieser Ausarbeitung soll eine Methode beschrieben werden, mit deren Hilfe man durch endlich viele Zufallszahlen 
eine ungefähr gleichmäßige Verteilung von Punkten im Raum generieren kann. Damit kann man sehr gut Daten simulieren, 
welche eine bestimmte Verteilung besitzen und, wie in der Natur üblich, keine harten Kanten besitzen oder komplett 
zufällig verteilt sind, sondern abfallen. Zu diesen Dichteverteilungen gehört zum Beispiel auch die Gaußschen Normalverteilung. 

Für die randomisierten Eingaben werden sogenannte "Golden Ratio Sequences", die von Schretter 
\cite{schretter-golden_ratio_sequences-2012} vorgestellt wurden, verwendet. Diese Punkte werden auf die von Chen 
und Asau \cite{chen_asau-generating_random_variates-1974} und Devroye \cite{devroye-non_uniform_random_variate-1986} 
beschriebene, verbesserte \hyperref[funktion]{Inversionsmethode} mit Hashing angewandt, wodurch das gewünschte Ergebnis 
erzielt wird. Diese Verfahrenskette wird unter \hyperref[impl]{Implementation} näher erläutert und durchgeführt. 


\subsection{Problemstellung}
Durch diese Ausarbeitung soll geklärt werden, wie die Hash-basierte Inversionsmethode funktioniert und ob  
sie bei der Generierung von multivariaten Verteilungen in höheren Dimensionen mithilfe von uniformen Dichten immer noch 
korrekt arbeitet und einsetzbar ist. Außerdem soll verglichen werden, ob sich der Einsatz der hier beschriebenen 
Methode als Ersatz von bereits existierenden anderen Methoden lohnt.

\begin{figure}
    \centering
    \includegraphics[width=.45\textwidth]{pdf_plots/grs_u500c}
    \caption{500 gleichmäßige, zufällige Punkte einer Goldenen-Schnitt-Sequenz \cite{schretter-golden_ratio_sequences-2012}.}
\end{figure}

\subsection{Stand der Technik}
Anstelle der hash-basierten Inversionsmethode kann auch ein Feld der Partialsummen der einzelnen Wahrscheinlichkeiten 
der zu generierenden Punkte verwendet werden. Auf dem Feld können dann unterschiedliche Methoden zum Finden der Werte 
verwendet werden. Dazu gehören lineare Suche von vorne nach hinten, binäre Suche oder ein Huffman-Baum.

Zur Generierung uniformer Zufallszahlen gibt es anstelle der Sequenzen des goldenen Schnittes auch Methoden wie
die von Halton \cite{wong_luk_heng-halton_points_sampling-1997}, Blue Noise \cite{yan-blue_noise_sampling-2015} oder Rang-1-Gitter 
\cite{prasad-rank1lattice-1973}. Doch da Schretter in seinem Paper zeigt, dass die von diesen Methoden generierten Zahlen 
nicht so gleichmäßig verteilt generiert werden, wie seine Vorgestellte \cite{schretter-golden_ratio_sequences-2012}, 
wird jene verwendet. 

\section{Funktionsweise}

\begin{frame}{Inversionsmethode}
	\begin{figure}
		\begin{subfigure}{.45\textwidth}
			\centering
			\includegraphics[width=\textwidth]{pdf_plots/logistic_-4to4.pdf}
            \caption{F mit Werten von $-4$ bis $4$}
		\end{subfigure}
		\begin{subfigure}{.45\textwidth}
			\centering
			\includegraphics[width=\textwidth]{pdf_plots/logistic_inv_0to1.pdf}
            \caption{F$^{-1}$ mit Werten von $0$ bis $1$}
		\end{subfigure}
	\end{figure}
\end{frame}

\begin{frame}{Inversionsmethode}
    \begin{itemize}
        \item Inverse einer Dichtefunktion benötigt 
        \item Entweder bekannt %\onslide<2->%, siehe \textit{\hyperref[tab:invFunctions]{Tabelle}} 
            oder numerisch integrierbar/annäherbar
    \end{itemize}
    \begin{table}%[htb!]
        \centering
        \begin{tabular}{l|l|l}
            Name         & Funktion & Zufällige Variable \\
            \hline\hline %& & \\
            Exponentiell & $1 - e^{-x}$ & $\log(1/U)$ \\ %\onslide<2->
            Logistisch   & $1 / (1 + e^{-x})$ & $-\log(\dfrac{1-U}{U})$ \\ %\onslide<3->
            Cauchy       & $1/2 + (1/\pi) \arctan(x)$ & $\tan(\pi U)$
        \end{tabular}
        \label{tab:invFunctions}
    \end{table}
\end{frame}

\begin{frame}{Hash-basierte Inversionsmethode}
    \begin{table}%[htb!]
        \centering
        \begin{tabular}{l|l|l|l}
            $X_j$   & F$(X_j)$  & $I_j$ & T$(I_j)$ \\
            \hline\hline % & & & \\
            $v_1$   & $0.021$   & $1$   & 1 \\ 
            $v_2$   & $0.021$   & $1$   & 1 \\
            $v_3$   & $0.021$   & $1$   & 1 \\
            $v_4$   & $0.021$   & $1$   & 1 \\ 
            $v_5$   & $0.021$   & $1$   & 1 \\
            $v_6$   & $0.021$   & $1$   & 1
        \end{tabular}
        \hspace{2em}
        \begin{tabular}{l|l|l|l}
            $X_j$   & F$(X_j)$  & $I_j$ & T$(I_j)$ \\
            \hline\hline % & & & \\
            $v_7$   & $0.021$   & $1$   & 1 \\ 
            $v_8$   & $0.021$   & $1$   & 1 \\
            $v_9$   & $0.021$   & $1$   & 1 \\
            $v_{10}$  & $0.021$   & $1$   & 1 \\ 
            $v_{11}$  & $0.021$   & $1$   & 1 \\
            $v_{12}$  & $0.021$   & $1$   & 1
        \end{tabular}
    \end{table}
\end{frame}

\section{Einschränkungen}
Nachdem die Funktionsweise der Hash-basierten Inversionsmethode erklärt wurde, 
werden nun mögliche Einschränkungen der Methode näher betrachtet und, wenn möglich, 
Lösungsansätze  aufgezeigt.


\subsection{Dimension}
Da in dem Abschnitt über die Funktionsweise der \hyperref[funktion]{Inversionsmethode} 
schon erklärt wurde, wie die Inversionsmethode in mehreren Dimensionen eingesetzt werden kann, 
ist leicht zu erkennen, dass mit etwas Mehraufwand auch die hash-basierte Methode in $n$ 
Dimensionen verwendet werden kann. Da die Werte der einzelnen Dimensionen rekursiv und 
abhängig voneinander berechnet werden, reicht nicht mehr nur eine Hashtabelle aus. Für 
jede Dimension muss eine Hashtabelle erstellt werden. Aufgrund der Abhängigkeit muss die 
Hashtabelle der höchsten Dimension zuerst berechnet werden. Dies funktioniert nach dem 
bekannten Muster, aber für alle weiteren Dimensionen müssen nun andere Funktionen zum 
Populieren der Tabellen verwendet werden. 

Anstelle der gegebenen Dichtefunktion $F$ mit einer Verteilung $\mathbb{P}(X) = p_X$ ist die 
jeweilige Verteilung gegeben durch $\mathbb{P}(X|Y)=\dfrac{\mathbb{P}(X\cap Y)}{\mathbb{P}(Y)}$, 
wobei $X$ der gesuchte Wert in der aktuellen Dimension ist und Y der Wert der höheren Dimension. 
An diesem Aufbau sieht man, dass die Hash-basierte Inversionsmethode zwar theoretisch für 
Punkt-Warping in beliebig vielen Dimensionen verwendet werden kann, doch bei zu vielen Dimensionen 
wird der Verwaltungsaufwand zu groß und die Initialisierung zu rechenaufwändig. Ab wievielen 
Dimensionen das der Fall ist, kann alllerdings nicht pauschal gesagt werden, da man mit kleineren 
Hashtabellen zwar ungenauer invertiert, aber weniger Rechenaufwand hat. 


\subsection{Abbildung}
Da die Inversionsmethode nur sinnvoll einsetzbar ist, wenn es möglich ist, das Inverse einer 
Funktion zu berechnen, oder die Dichtefunktion direkt bekannt ist, ist auch die hashbasierte 
Inversion nur dann einsetzbar. Ansonsten erhält die Methode alle Eigenschaften ihrer 
Eingabefunktion. Dazu gehören Winkelerhalt, aber auch Korrelation und statistische 
Unabhängigkeit, da durch die Hashtabelle keine Eigenschaften verloren gehen, solange 
die Punkte nicht zu dicht beieinander liegen (siehe \hyperref[Dichte]{Dichte}).


\subsection{Dichte}
\label{Dichte}
Da durch den Einsatz der Hashtabelle nur endlich viele Werte gespeichert werden können, ist 
eine Berechnung von Zufallsvariablen $X$ nur dann sinnvoll, wenn diese nicht zu dicht 
beieinander liegen und durch Annäherung auf einen Punkt gemappt werden würden, da die Tabelle 
eine zu geringe Größe hat, um ausreichend viele unterschiedliche Werte zu speichern. Deshalb 
sollte vor Einsatz der Tabelle die Dichte bekannt sein, damit die Größe $N$ der 
Hashtabelle optimal gewählt werden kann.

\section{Vergleich}

\begin{frame}{Vergleich der Invertierungsmethoden}
    \begin{figure}
        \centering
        \includegraphics[width=.7\textwidth]{Screenshots/chenAsauHashTableComp.pdf}
        \caption{\cite{chen_asau-generating_random_variates-1974}}
    \end{figure}
\end{frame}



\section{Implementierung}
\label{impl}
\begin{figure*}[htb!]
    \centering
    \begin{subfigure}[b]{.3\textwidth}
        \centering
        \includegraphics[width=\textwidth]{pdf_plots/idf2_log-sin_iu500c-RND.pdf}
        \caption{Eingabewerte zufällig generiert mit \code{random}}
        \label{fig:logsin_random}
    \end{subfigure}
    \hfill
    \begin{subfigure}[b]{.3\textwidth}
        \centering
        \includegraphics[width=\textwidth]{pdf_plots/idf2_log-sin_iu500c-GRS.pdf}
        \caption{Eingabewerte aus einer Goldenen Schnitt Sequenz}
        \label{fig:logsin_uniform}
    \end{subfigure}
    \caption{$500$ Punkte mit einer logistischen Dichte auf der X- und einer sinusoiden auf der Y-Achse}
    \label{fig:rand_vs_uniform}
\end{figure*}


Um einen Eindruck der hashbasierten Inversionsmethode und ihrer Arbeitsweise zu bekommen, 
wird jetzt eine konkrete Implementierung dieser angesprochen und Ergebnisse bei 
unterschiedlichen Eingaben gezeigt. Geschrieben wurde das Programm in Python mithilfe der 
Pakete \textit{numpy} und \textit{matplotlib}.

Schretter \cite{schretter-golden_ratio_sequences-2012} stellt am Ende der Arbeit eine 
C++-Methode zur Generierung einer zweidimensionalen Goldenen-Schnitt-Sequenz zur Verfügung. 
Diese Methode wurde in ein Python-Script übertragen und alle, in diesem Kapitel erwähnte, zufällige, uniformen
Variablen $U$ werden mit diesem Script generiert.

Zum Testen der vorgestellten Methode wurde das Konzept von Chen und Asau \cite{chen_asau-generating_random_variates-1974} 
implementiert. Dabei wurde kein großer Wert auf optimale Laufzeiten gelegt, sondern darauf, eine Demonstration 
des Verfahrens geben zu können.

% TODO
% Ergebnisse bei unterschiedlichen Werten auswerten nicht nur im Bild
\dots




\section{Zusammenfassung}

\begin{frame}{Kurz \& Knapp}    
    \begin{itemize}
        \item Schnelle Generierung von großen Datenmengen
        \item Beibehaltung aller Eigenschaften der ursprünglichen Funktion
        \item einfaches Prinzip
    \end{itemize}
\end{frame}

\begin{frame}{Ausblick}   

\end{frame}

% Literaturverzeichnis in Literatur.bib
\bibliographystyle{plain}
\bibliography{Literatur}

% Code
\newpage % TODO keep it??
\section{Code}

\begin{algorithm}
    \caption{Implementation of a guide table in Python}
    \begin{algorithmic}
        \Statex \Comment{T is the table, PS, the partial sums, N the size of the table}
        \Function{\_\_init\_\_}{self, N}
            \State self.T = np.array([])
            \State self.PS = np.array([])
            \State self.N = N
        \EndFunction

        \Statex \Comment{func is the density function used}
        \Function{setup\_table}{self, func}
            \For{i in range(0, self.N)}
                \State self.append(self.T, func(i / self.N))
            \EndFor
        \EndFunction
        \State
        \Function{setup\_partial\_sums}{self, partial\_sums}
            \State self.PS = np.array(partial\_sums)
        \EndFunction
        \State
        \Function{hash}{self, U}
            \State return self.N * U
        \EndFunction
        
        \Comment{test until inequation \eqref{eq:hash_ineq} is true}
        \Function{search}{self, U}
            \State Z = self.hash(U)
            \While{self.PS[Z] <= U}
                \State Z += 1
            \EndWhile
            \State return self.PS[Z]
        \EndFunction
    \end{algorithmic}
\end{algorithm}


\end{document}