\section{Zusammenfassung}
Mit der vorgestellten Methode können vor allem große Datenmengen sehr schnell generiert werden. 
Solange die Daten nicht zu dicht beieinander liegen und die Größe der Hashtabelle gut genug gewählt 
ist, werden sogar alle Eigenschaften der Funktion behalten. Dadurch kann man die hashbasierte 
Inversiosmethode mit Modellen zur uniforem Generierung von Zufallszahlen, wie zum Beispiel Golden 
Ratio Sequences \cite{schretter-golden_ratio_sequences-2012} oder Fibonnaci Grids 
\cite{frisch_hanebeck-deterministic_gaussian_sampling-2021}, verbunden werden. Da nur minimaler 
Verwaltungsaufwand betrieben werden muss, um die Inversionsmethode in höheren Dimensionen anzuwenden, 
kann sie beinahe überall eingesetzt werden, um in $O(1)$ Zufallszahlen zu generieren. Durch die 
einfache Arbeitsweise, die in dieser Ausarbeitung beschrieben wurde, kann die vorgestellte Methode 
von jedem implementiert und verwendet werden. 


\subsection{Verbesserungsmöglichkeiten}
Wie in dem Abschnitt \hyperref[Dichte]{Dichte} angesprochen, kann die Methode verbessert werden, 
in dem nicht nur gleichverteilte Werte in der Hashtabelle gespeichert werden, sondern eine 
Vorauswahl getroffen wird, die für Gebiete mit hoher Dichte mehr Einträge in der Hashtabelle speichert.
