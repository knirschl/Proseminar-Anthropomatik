\section{Implementierung}
\label{impl}
\begin{figure*}%{htb!}
    \centering
    \begin{subfigure}[b]{.3\textwidth}
        \centering
        \includegraphics[width=\textwidth]{pdf_plots/idf2_log-log_iu500c-GRS.pdf}
        \caption{Logistische Dichte auf der X- und Y-Achse}
        \label{fig:loglog_examplePlot}
    \end{subfigure}
    \hfill
    \begin{subfigure}[b]{.3\textwidth}
        \centering
        \includegraphics[width=\textwidth]{pdf_plots/idf2_sin-expo_iu500c-GRS.pdf}
        \caption{Sinusodial auf der X- und exponentiell auf der Y-Achse verteilt}
        \label{fig:sinexpo_examplePlot}
    \end{subfigure}
    \hfill
    \begin{subfigure}[b]{.3\textwidth}
        \centering
        \includegraphics[width=\textwidth]{pdf_plots/idf2_snrv_iu500c-GRS.pdf}
        \caption{Standardnormalverteilung in beide Richtungen}
        \label{fig:snrv_examplePlot}
    \end{subfigure}
    \caption{Mehrere inverse Funktionen mit je $500$ Punkten.}
    \label{fig:examplePlot}
\end{figure*}

Um einen Eindruck der hashbasierten Inversionsmethode und ihrer Arbeitsweise zu 
bekommen, wird jetzt eine konkrete Implementierung dieser angesprochen und E
rgebnisse bei unterschiedlichen Eingaben gezeigt. Geschrieben wurde das Programm 
in Python mithilfe der Pakete \textit{numpy} und \textit{matplotlib}.

Schretter \cite{schretter-golden_ratio_sequences-2012} stellt am Ende der Arbeit 
eine C++-Methode zur Generierung einer zweidimensionalen Goldenen-Schnitt-Sequenz 
zur Verfügung. Diese Methode wurde in ein Python-Script übertragen und alle, in 
diesem Kapitel erwähnte, zufällige, uniformenVariablen $U$ werden mit diesem Script 
generiert.

Zum Testen der vorgestellten Methode wurde das Konzept von Devroye \cite{devroye-non_uniform_random_variate-1986} 
implementiert. Dabei wurde kein großer Wert auf optimale Laufzeiten gelegt, sondern 
darauf, eine Demonstration des Verfahrens geben zu können. Am Ende dieser Arbeit \ref{alg:inv_hash1}
ist in einer Mischung aus Python und Pseudocode der Kern der Implementierung zu 
sehen. Zusätzlich dazu ist der gesamte Algorithmus in diesem \href{https:github.com/knirschl/Proseminar-Anthropomatik}
{Git-Repository}zu finden. 

Die eigentliche Inversionsmethode mit Hashtabelle ist in \code{guidetable.py} 
implementiert. Dabei beschreiben die Methoden \code{\_\_init\_\_}, \code{hash} und \code{search\_single} 
das Invertieren und Hashen.. In \code{\_\_init\_\_} wird die Hashtabelle mit ihren 
Werten populiert, wie es in dem Kapitel \hyperref[funktion_invhashing]{Inversion mit Hashing} 
beschrieben wurde. Die Funktion \code{search\_single} sucht ab dem gehashten Wert der 
eingegebenen Zufallsvariable solange in der Partialsummenliste, die davor entweder berechnet 
oder übergeben wurde, bis die Ungleichung zum Finden des zugehörigen $X$-Wertes \eqref{eq:hash_ineq} 
erfüllt ist. Damit nicht jeder Punkt einzeln berechnet werden muss, kann mit der Funktion 
\code{search} ein zweidimensionales Feld übergeben werden, für das dann die korrespondierenden 
Werte berechnet und zurückgegeben werden. Eine andere wichtige Datei ist \code{golden\_ratio\_sequence.py}. 
Diese Datei fasst Methoden zur Generierung von Goldenen-Schnitt-Sequenzen zusammen und wurde 
von Schretter übernommen und nach Python übertragen. 

In \ref{fig:examplePlot} sind drei Punktemengen zu sehen, bei denen eine Goldene-Schnitt-Sequenz 
in der vorgestellten Implemtierung generiert und dann mit \code{guidetable.py} verformt wurden. 
Alle drei Abbildungen zeigen 500 Punkte, wobei 
einmal \ref{fig:loglog_examplePlot} auf beiden Achsen eine logistische Dichte 
\ref{tab:invFuncs} angewendet wurde. Im Gegensatz dazu ist bei 
\ref{fig:sinexpo_examplePlot} die X-Koordinate mit einer Sinusfunktion verschoben 
und die Y-Koordinate nach einer exponentiellen Verteilung \ref{tab:invFuncs}. Man 
kann leicht die sinusförmigen Wellen sehen, bei denen die Punkte enger aneinander 
liegen. In der dritten Abbildung \ref{fig:snrv_examplePlot} sind beide 
Koordinatenanteile standardnormalverteiltverformt, wodurch das bekannte Spiralmuster 
entsteht. 